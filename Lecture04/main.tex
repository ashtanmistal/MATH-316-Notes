\documentclass{article}

\usepackage{../preamble}
\standalonetrue

\title{MATH 316 Lecture 4}
\author{Ashtan Mistal}
\date{May 18 2021}

\begin{document}

\ifstandalone
\maketitle
\fi

\graphicspath{{./Lecture04/}}

\section{Introduction}

\subsection{Week 2:}

\begin{itemize}
    \item finish power series
    \item Bessel's function
    \item Intro to PDEs
\end{itemize}

\subsection{Week 3:}

\begin{itemize}
    \item Fourier series and separation of variables
    \item Heat equations
    \item Wave equations
    \item Laplace equations
\end{itemize}

\subsection{Week 4:}

\begin{itemize}
    \item Boundary value problems and Sturm-Louiville Theory
    \item Eigenfunctions and eigenvalues
    \item Sturm-Louisvill theory for BVP
    \item Non-homogenous boundary value problems
\end{itemize}

\subsection{Week 5:}

\begin{itemize}
    \item Numerical methods for solving PDEs
\end{itemize}

\section{Recap of Previous Week}

What we're covering next was partly covered in previous lectures by Parisa:

\hfill

Series Solutions near a regular singular point (RSP)

$$ P(x) y'' + Q(x) y' + R(x)y = 0$$

We consider an expansion about a regular singular point $x_0$ of the ODE. 

Next, we take the limits:

$$\lim_{x \to x_0} (x - x_0) \frac{Q(x)}{P(x)} = p_0$$

$$\lim_{x \to x_0} (x - x_0)^2 \frac{R(x)}{P(x)} = q_0$$

When these two limits exist and are finite, then this leads to the following. We divide $ P(x) y'' + Q(x) y' + R(x)y = 0$ by $P(x)$ and multiply it by $(x - x_0)^2$:

$$(x-x_0)^2 y'' + (x - x_0) \underbrace{\left\{(x - x_0) \frac{Q(x)}{P(x)} \right\}}_{p(x) = p_0 + p_1 (x - x_0) + ...} y' + \underbrace{\left\{ (x - x_0)^2 \frac{R(x)}{P(x)} \right\}}_{q(x) = q_0 + q_1 (x - x_0) + ...} y = 0$$


$$(x - x_0)^2 + p_0 (x - x_0) y' + q_0 y = 0$$

is the C-E equation. This equation has a solution of the following format:

$$y_0(x) = (x - x_0)^r$$

To include the effect of the neglected terms, we modify the above solution:
$$y(x) = \underbrace{(x - x_0)^r}_{\text{C-E solution}} \sum_{n = 0}^{\infty} \underbrace{a_n (x - x_0)^n}_{\text{The solution}}$$


with $a_0 \neq 0$. (2) is the modified equation for $y_0(x) = (x - x_0)^r$ and is  Frobenius series. 

\subsection{Solution procedure}

\begin{enumerate}
    \item Find values of r
    \item Find the recursive relation for n
    \item Find the radius of convergene for $\sum_{n = 0}^{\infty} a_n (x - x_0)^{n +r}$
\end{enumerate}

\subsection{Example 1}

$$Ly = 2x^2 y'' - x y' + (1-x)y = 0$$

Singular point here is when $x_0 = 0$

Again, to test if it is a singular point, we take the limits. 

$$p_0 = \lim_{x \to 0} \frac{-x}{2x^2} x = \frac{-1}{2}$$

Similarly, we check $q_0$:

$$q_0 = \lim_{x \to 0} \frac{(1-x)}{2x^2} x^2 = \frac{1}{2} = q_0$$

Hence, the singular point is a RSP. 

$$\Rightarrow y(x) = x^r \sum_{n = 0}^\infty a_n x^n$$

Step 1: Find values of r

Characteristic equation: $r(r-1) + p_0 r + q_0 = 0$

This becomes $r(r-1) - \frac{r}{2} + \frac{1}{2} = 0 \Rightarrow r^2 - \frac{3}{2} r + \frac{1}{2} = 0$

Hence, $r = 1$ and $r = \frac{1}{2}$ are roots of the indicial equation. 

$$y_0 = c_1 x + c_2 x^{\frac{1}{2}}$$

Now, we go to step 2: Find the recursive relation for $n$. 

Take the first and second derivative of the summation (that we've used before) and sub into the equation. 

$$2\sum_{n = 0}^{\infty} a_n (n+r)(n+r-1) x^{n+r} - \sum_{n = 0}^{\infty} a_n (n+r) x^{n+r} + \sum_{n = 0}^\infty a_n x^{n+r} - \sum_{n = 0}^\infty a_n x^{n+r+1} = 0$$

Changing index:

$$2\sum_{n = 0}^{\infty} a_n (n+r)(n+r-1) x^{n+r} - \sum_{n = 0}^{\infty} a_n (n+r) x^{n+r} + \sum_{n = 0}^\infty a_n x^{n+r} - \underbrace{\sum_{n = 0}^\infty a_n x^{n+r+1}}_{m = n+1} = 0$$

All others are $m = n$. 

With that, we get the following:

$$2\sum_{n = 0}^{\infty} a_n (n+r)(n+r-1) x^{n+r} - \sum_{n = 0}^{\infty} a_n (n+r) x^{n+r} + \sum_{n = 0}^\infty a_n x^{n+r} - \sum_{n = 1}^\infty a_{n-1} x^{n+r}$$
(Removing $=0$ due to space)

Peeling off the first terms ($n=0$), we end up with the following sum:

$$\underbrace{2 a_n r(r-1)x^r - a_0 r x^r + a_0 x^r}_{ = a_0 x^r \left( 2r^2 - 3r + 1 \right)} \longrightarrow ...$$
$$\hookrightarrow + \sum_{n = 1}^\infty \left[ 2 a_n (n+r) (n+r-1) - a_n (n+r) + a_{n-1} \right] x^{n+r} = 0$$

Hence, $a_n = \frac{a_{n-1}}{(n+r)(2(n+r)-3) + 1}$ is the recursive relation for r values

Finding the recursive relation for $r_1$ and $r_1$:

$$r_1 \Rightarrow a_n = \frac{a_{n-1}}{(n+1)(2(n+1)-3)+1} = \frac{1_{n-1}}{2n^2 + n}$$

$n = 1: a_1 = \frac{a_0}{3}; n = 2: a_2 = \frac{a_1}{10} = \frac{a_0}{30}; n = 3: a_3 = \frac{a_2}{21} = \frac{a_0}{630}$

Therefore:

$$y(x) = a_0 x^1 (1 + \frac{x}{3} + \frac{x^2}{30} + \frac{x^3}{630} + ...)$$

Next, for $r_1 = \frac{1}{2}$:

$$a_n  =\frac{a_{n-1}}{(n + \frac{1}{2} (2n+1-3) + 1} = \frac{a_{n-1}}{2(n + \frac{1}{2})(n-1) + 1}$$

$n = 1: a_1 = \frac{a_0}{1}; n = 2:  a_2 = \frac{a_1}{6} = \frac{a_0}{6}; n = 3: a_3 = \frac{a_0}{90}$

So, we have the following:

$$y_2(x) = a_0 x^{\frac{1}{2}} \left[ 1 + x + \frac{x^2}{6} + \frac{x^3}{90} + ... \right]$$

Now, onto step 3:

Find the radius of convergence. 

To find the radius of covergence, we use the ratio test:

$$\lim_{n \to \infty} | \frac{a_n x^n}{a_{n-1}x^{n-1}}|$$

$$ = \lim_{n \to \infty} | \frac{a_{n-1} x^n}{(n+r) (2(n+r)-3) + 1} \frac{1}{a_{n-1} x^{n-1}} |$$

\textbf{The above is not correct. Why?}

$$\lim_{n \to \infty} |x| |\frac{1}{(n+r) (2(n+r)-3)+1}| = 0$$

$\Rightarrow p = \infty$ for all $x$ values

Final solution:

$$y(x) = y_1(x) + y_2(x)$$

$$y(x) = C_1 x \left(1 + \frac{x}{3} + \frac{x^2}{30} + \frac{x^3}{630} + ... \right) + C_2 x^\frac{1}{2} \left(1+x+\frac{x^2}{6} + \frac{x^3}{90} + ... \right)$$


\section{Bessel's equation}

Applications: PDEs on circular / cylindrical domain. 

e.g. heating and cooling in circular / cylindrical geometries, i.e. pipes and heat exchangers. 

\hfill

The equation format:
\begin{equation}
    \label{Bessel's Equation}
    Ly = x^2 y'' + xy' + \left(x^2 - \nu^2 \right) y = 0
\end{equation}

$\nu$ is a constant and specifies the order of the equation. 

$x_0 = 0$ is a regular singular point, since $\lim_{x \to 0} \frac{x}{x^2} (x) = 0 = p_0$ and $\lim_{x \to 0} \frac{x^2 - \nu^2}{x^2}(x^2) = - \nu^2 = q_0$

\subsection{Step 1: Finding the $r$ values}

The characteristic equation is:

$$r(r-1) + p_0 r + q_0 = 0$$

$$r(r-1) + r - \nu^2 = 0 \Rightarrow r = \pm \nu$$

\subsection{Step 2: Frobenius part -- Finding recursive relation}

$$y = \sum_{n = 0}^\infty a_n x^{n+r}$$

$$y' = \sum_{n =0}^{\infty} a_n (n+r)x^{n+r-1}$$

$$y'' = \sum_{n=0}^{\infty} a_n (n+r) (n+r-1) x^{n+r-2}$$

Substituting into the equation, we get the following:

$$Ly = x^2 y'' + xy' + \left(x^2 - \nu^2 \right) y = 0$$

$$Ly = x^2 \sum_{n=0}^{\infty} a_n (n+r) (n+r-1) x^{n+r-2} \rightarrow ...$$

$$ \hookrightarrow + x\sum_{n =0}^{\infty} a_n (n+r)x^{n+r-1} \rightarrow ...$$
$$ \hookrightarrow + \left(x^2 - \nu^2 \right) \sum_{n = 0}^\infty a_n x^{n+r} = 0$$

Shifting index:

$$Ly = x^2 y'' + xy' + \left(x^2 - \nu^2 \right) y = 0$$

$$Ly = x^2 \sum_{n=0}^{\infty} a_n (n+r) (n+r-1) x^{n+r-2} \rightarrow ...$$

$$ \hookrightarrow + x\sum_{n =0}^{\infty} a_n (n+r)x^{n+r-1} \rightarrow ...$$
$$ \hookrightarrow + \underbrace{x^2\sum_{n = 0}^\infty a_n x^{n+r+2}}_{m = n+2} - \nu^2\sum_{n = 0}^\infty a_n x^{n+r} = 0$$

Hence, we get the following:

$$Ly = \sum_{n = 0}^\infty a_n (n+r)(n+r-1) x^{n+r} + \sum_{n=0}^\infty a_n (n+r) x^{n+r} + \sum_{n = 2}^\infty a_{n-2} x^{n+r}$$

Next, we peel off the first two terms to match the summation. 

$$a_0 (r(r-1) + r - \nu^2) x^r + a_1 ((1+r) r + (1+r) - \nu^2) x^{r+1} \rightarrow ... $$
$$ \hookrightarrow + \sum_{n = 2}^\infty a_n \left[ a_n \left( (n+r)^2 - \nu^2 \right) + a_{n-2} \right] x^{n+r} = 0$$

$$a_0 x^r (r^2 - \nu^2) + a_1 x^{r+1} \left((r+1)^2 - \nu^2 \right)  + \sum_{n = 2}^\infty a_n \left[ a_n \left( (n+r)^2 - \nu^2 \right) + a_{n-2} \right] x^{n+r} = 0$$

\end{document}
